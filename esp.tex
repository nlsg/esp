% Created 2022-05-28 Sat 20:56
% Intended LaTeX compiler: pdflatex
\documentclass[11pt]{article}
\usepackage[utf8]{inputenc}
\usepackage[T1]{fontenc}
\usepackage{graphicx}
\usepackage{longtable}
\usepackage{wrapfig}
\usepackage{rotating}
\usepackage[normalem]{ulem}
\usepackage{amsmath}
\usepackage{amssymb}
\usepackage{capt-of}
\usepackage{hyperref}
\author{John Doe}
\date{\today}
\title{esp}
\hypersetup{
 pdfauthor={John Doe},
 pdftitle={esp},
 pdfkeywords={},
 pdfsubject={},
 pdfcreator={Emacs 27.2 (Org mode 9.6)}, 
 pdflang={English}}
\begin{document}

\maketitle
\tableofcontents

[NAME] is a web based dashboard running on a esp32,
with it, you can control all I/O peripherals plus read some
metadata from the esp
\section{{\bfseries\sffamily TODO} separate project agnositc stuff}
\label{sec:org0417e50}
\section{{\bfseries\sffamily TODO} fix user wheel group problem}
\label{sec:orgcc8e3c0}
\section{{\bfseries\sffamily TODO} ideas}
\label{sec:org1cc9a29}
\subsection{try to let gopro (cam) interface with esp}
\label{sec:orgf76b5bd}
\section{file-gen}
\label{sec:org601a6e9}
\begin{verbatim}
from os import popen
"""this script outputs an emacs-org table (simple ascii table with '-' and '|'),
    which is programmatically aggregated.
    Since this is meant to be used in an org-file,
    there is a return statement in the global scope"""

src = "./src/"              # src directory from where files are grabbed
url = "192.168.43.31:/"     # ws domain where files are transferred [col 'esp']
files = popen(f"ls {src}").read().split("\n")[:-1]

webrepl_cli_bin = "/home/$USER/py/esp/webrepl/webrepl_cli.py" # [col 'esp']

pop = lambda c: popen(c).read()[:-1]  # because it sucks to always read, especially with the last '\n'
tab = lambda l: "|" + "|".join(l) + "|"   # outputs the table rows -> |x|y|z|
emacs_link = lambda t,l,n: f"[[{t}:{l}][{n}]]"
f_link = lambda l,n: emacs_link("file", l, n)
s_link = lambda l,n: emacs_link("shell", l.split("&")[0] + " &", n)

rows = (
    {"head": "files",
     "body": lambda f: f_link(src + f, f),
     "tail": f"[[file:{src}][{src}]]"},
    {"head": "esp",
     "body": lambda f: s_link(f"{webrepl_cli_bin} {src}{f} {url}{f}", "->"),
     "tail": ""},
    {"head": "du",
     "body": lambda f: pop(f"du -h {src}{f}").split()[0],
     "tail": pop(f"du -S -h {src}").split()[0]},
    {"head": "wc",
     "body": lambda f: ", ".join(pop(f"cat {src}{f} |  wc ").split()),
     "tail": ", ".join(pop(f"cat {src}* |  wc ").split())})

def table_as_string():
    return "\n".join(
        (tab(c["head"] for c in rows),
        "|-",
        "\n".join(tab([c["body"](f) for c in rows]) for f in files),
        tab(c["tail"] for c in rows)))

if __name__ == "__main__":
    return table_as_string()  # for emacs-org src_blocks
    # print(table_as_string() # to not get en error otherwise
\end{verbatim}
\begin{verbatim}
import uasyncio; from api import coro
\end{verbatim}

\section{hardware}
\label{sec:orgd05751d}
\subsection{esp32}
\label{sec:orgf959c51}
\subsubsection{pinout}
\label{sec:orgb37466e}
\href{./assets/pinout-0.png}{left-side}
\href{./assets/pinout-1.png}{right-side}
\href{./assets/pinout-disc.png}{discription}
\begin{enumerate}
\item legacy pinouts
\label{sec:orge543765}
\href{./assets/esp32-pinout.png}{pinout}
\href{./assets/esp32-pinout1.png}{pinout\_}
\end{enumerate}
\subsubsection{block-diagram}
\label{sec:orgd91cbb9}
\href{./assets/esp32-block-diagram.png}{block-diagram}
\subsubsection{hacks}
\label{sec:org0af544a}
\href{https://microcontrollerslab.com/esp32-esp8266-adc-micropython-measure-analog-readings/}{make adc resolve 12bit}
\texttt{adc.atten(ADC.ATTN\_11DB)       \#3.3V full range of voltage}
\section{API references}
\label{sec:orge563328}
\subsection{\href{https://docs.micropython.org/en/latest/esp32/quickref.html}{micropython implementation of espressif}}
\label{sec:org0ced470}
\href{https://docs.micropython.org/en/latest/library/uasyncio.html}{uasyncio}
\subsection{\href{https://microdot.readthedocs.io/en/latest/api.html}{Microdot}}
\label{sec:org9c513af}
\href{https://bhave.sh/micropython-microdot/}{Tut of Microdot}
\section{api\textsubscript{test.py}}
\label{sec:org91da3da}
\begin{verbatim}
#!/usr/bin/env python3

"""this script for the [NAME] api"""

import requests
from time import perf_counter

ROOT_URL = "http://192.168.43.31:5000/"

get_gen = lambda calls: (requests.get(ROOT_URL + call["route"], params=call["payload"]).json()
                         for call in calls)

def benchmark(gen, count):
    t1 = perf_counter()
    return {"results": [[requests.get(ROOT_URL + call["route"], params=call["payload"])
                                .json()
                        for call in api_calls] for i in range(count)],
            "time": (t := perf_counter() - t1),
            "count": count,
            "average_ms": t / count,
            "request/s": 1 / ( t / count )}

def pop_keys(keys, *dicts):
    """removes all keys in dicts and returns a list of the resulting dicts"""
    res_dict = []
    for b in dicts:
        [b.pop(k) for k in keys]
        res_dict.append(b)
    return res_dict

api_calls = ({"route": "api/gpio/set",
             "payload": {"pin":26}}, )

# gpio_bench = pop_keys((), benchmark(get_gen(api_calls), 1))
gpio_bench = benchmark(get_gen(api_calls), 1)

for report in gpio_bench:
    for k in report:
        print(k, " -> ", report[k])

\end{verbatim}

\section{curl (test API)}
\label{sec:org4775819}
\hyperref[sec:org9007b89]{api spec}
\subsubsection{Root / Doc}
\label{sec:orgf9f8c06}
\href{curl 192.168.43.31:5000/ \&}{/  <-}
\href{curl -G -d "pin=2" 192.168.43.31:5000/api/gpio \&}{/gpio <-}
\subsubsection{Digital}
\label{sec:orgf0ad521}
-> \href{curl -G -d "pin=2" 192.168.43.31:5000/api/gpio/get \&}{/gpio/get 2}
-> \url{curl -G -d "pin=33" 192.168.43.31:5000/api/gpio/get \&}
-> \href{curl -G -d "pin=two" 192.168.43.31:5000/api/gpio/get \&}{/gpio/get type error str instead of int}
-> \url{curl -G -d "pin=26" 192.168.43.31:5000/api/gpio/set \&}
-> \href{curl -G -d "pin=2" 192.168.43.31:5000/api/gpio/set \&}{/gpio/set 2 toggle}
-> \href{curl -G -d "pin=2" -d "value=0" 192.168.43.31:5000/api/gpio/set \&}{/gpio/set 2 OFF}
-> \href{curl -G -d "pin=2" -d "value=1" 192.168.43.31:5000/api/gpio/set \&}{/gpio/set 2 ON}
-> \href{curl -G -d "pin=two" -d "value=1" 192.168.43.31:5000/api/gpio/set \&}{/gpio/set 2 type error str instead of int}
\subsubsection{Analog}
\label{sec:orgb3ec042}
\href{curl -G -d "pin=33"  192.168.43.31:5000/api/gpio/read \&}{/gpio/read 33 <-}
\href{curl -G -d "pin=two" 192.168.43.31:5000/api/gpio/read \&}{/gpio/read 33 type error str instead of int <-}
\subsubsection{ds18b20}
\label{sec:orgb040303}
\href{curl -G -d "pin=4"  192.168.43.31:5000/api/gpio/ds18 \&}{/gpio/ds18 4 <-}

\section{tools}
\label{sec:org54f83e3}
\subsection{{\bfseries\sffamily TODO} document img flashin process}
\label{sec:orgb2f6450}
\subsection{{\bfseries\sffamily TODO} cli\textsubscript{socket.py}}
\label{sec:org86a04fc}
\begin{verbatim}
import websockets
#!/usr/bin/env python

import asyncio
import websockets

uri = "ws://192.168.43.31:8266"

async def hello():
    msg = ""
    async with websockets.connect(uri) as websocket:
        await websocket.send(msg)
        msg = input(">>>" + await websocket.recv())

asyncio.run(hello())
\end{verbatim}
\subsection{doc extractor (docer.py)}
\label{sec:orgfa380a2}
\begin{verbatim}
#!/usr/bin/env python
"""this python script extracts signature and docstring, of given functions.
Recogniseing decoratores is yet not implemented.
The data is then outputed to a json or org file,
later is especially usefull, if this script is
implemented in a org file. In that case, make
sure that the src_block has the ':results raw' option added
"""

from os import popen
from json import dump
from datetime import datetime as dt

SRC_PATH = "./src/"
OUT_FILE = "./docs.json"

files = [SRC_PATH + f for f in popen("ls ./src").read().split("\n") if ".py" in f]

def extract_docs(files):
    """outputs a dict with given filenames as keys,
    whose values are dicts with function signature
    and docstring key value pairs"""
    docs = {}
    for file in files:
        functions = {}
        for f in "".join(open(file, "r").readlines()).split("def "):
            try:
                doc = f.split('"""')[1]
            except IndexError:
                doc = "N/A"
            functions.update({f.split("\n")[0]: doc})
        docs.update({file: functions})
    return docs

def dict_to_org(d, indent=1):
    """recursivly transforms a dict to an org-file.
    The heading level is controled by the indent parameter
    e.g.
    >>> dict_to_org({"foo":"bar","bar"{"oof":"baz"}}, indent=2)"""
    # ** foo
    #    bar
    # ** bar
    # *** off
    #     baz
    return "\n".join([
        "\n".join((
            "*"*(indent) + " " + str(k),
            dict_to_org(d[k], indent+1)
            if isinstance(d[k], dict)
            else str(d[k])))
        for k in d])

return "** -> docstrings (" \
        + dt.now().strftime("%Y-%m-%d - %H:%M:%S") \
        + ")\n" \
        + dict_to_org(extract_docs(files), indent=3)

# dump(docs, open(OUT_FILE, "w"), indent=2)
\end{verbatim}
\subsection{rshell (filesystem)}
\label{sec:org4044702}
rshell provides a shell which mounts the
filesystem of the microcontroller (eps32)
under the path /pyboard
\texttt{rshell -p /dev/ttyUSB0 -b 115200}

rshell also supports scripts (-f option)
edit and tangel this rshell script block,
\begin{verbatim}
cp /home/nls/py/esp/src/boot.py /pyboard
\end{verbatim}
and execute it
\url{sudo rshell -p /dev/ttyUSB0 -b 115200 -f \~/py/esp/dist.rshell \&} <-
\subsection{webrepl\textsubscript{cli} (hotswap code)}
\label{sec:orgabadcfb}
transfer a file with webrepl (don't forget to save the file beforhand)
-> \href{ /home/\$USER/py/esp/webrepl/webrepl\_cli.py -p "" \~/py/esp/src/gpio.py 192.168.43.31:/gpio.py \&}{gpio}
-> \href{ /home/\$USER/py/esp/webrepl/webrepl\_cli.py -p "" \~/py/esp/src/main.py 192.168.43.31:/main.py \&}{main}

\subsection{screen (connect to console/repl)}
\label{sec:org2247898}
C-c to abort the programm and access
\texttt{screen /dev/ttyUSB0 115200}

\subsection{\href{file:///home/nls/py/esp/webrepl/webrepl.html}{webrepl.html}}
\label{sec:org37adaf0}

\section{specification}
\label{sec:orgbfb2e7b}
\subsection{{\bfseries\sffamily TODO} ULP API}
\label{sec:org785fdae}
\subsection{I/Os}
\label{sec:orge2a5d2e}
\begin{center}
\begin{tabular}{llll}
type & location & porpuse & pin\\
\hline
INPUTS &  &  & \\
tcp/ip ui & portable &  & \\
dht22 & lung &  & \\
dht22 & bloom &  & \\
dht22 &  &  & \\
humid & soil / bloom &  & \\
co2 & bloom &  & \\
pt100 & outdoor &  & \\
\hline
OUTPUTS &  &  & \\
light & bloom &  & \\
light & veg &  & \\
circ & bloom &  & \\
circ & vig &  & \\
exhaust &  &  & \\
mister & lung &  & \\
dosage-punp & feed &  & \\
\end{tabular}
\end{center}

\subsection{Control-Loops}
\label{sec:org176c75d}
\begin{center}
\begin{tabular}{ll}
actor & sensor\\
light & time\\
mister & humid, gro\\
exhaust & temperature\\
\end{tabular}
\end{center}

\subsection{UI features}
\label{sec:org1fc4abb}
\begin{itemize}
\item E-Stop (Mute)
\item abstract mapping
\end{itemize}
\subsection{API}
\label{sec:org9007b89}
\hyperref[sec:org4775819]{test the api with curl}
\href{cat \~/py/esp/docs.json|sed 's,\\":,\\n\\t\\t,'|sed  's,\{,,'|sed 's,\},,'|sed 's,\\",,'|sed 's.,..' | sed 's,\\\\n,\\n,' \&}{read            <- docs.json}
\hyperref[sec:orgfa380a2]{generate docs   -> docs.json}
\begin{verbatim}
return "\n".join([l for l in open("./src/api.py", "r").readlines() if l[0] == "@"])
\end{verbatim}

\subsection{sensors}
\label{sec:org54d714c}
\subsubsection{pt100}
\label{sec:org38dfa17}
desired resistor for pt100: \textasciitilde{}2.57kOhm
\subsubsection{OEM DS18B20}
\label{sec:org5afc2f3}
\href{https://randomnerdtutorials.com/micropython-ds18b20-esp32-esp8266/}{tutorial for esp32}
\section{merge org files}
\label{sec:org844e627}
\begin{verbatim}
FILE=~/base.org
echo "* -> "$FILE
cat $FILE | sed 's,[*] ,** ,'
\end{verbatim}
\end{document}
